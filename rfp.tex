Monocle Labs requests funding from the Defense Advanced Research Projects
Agency's (DARPA) Defense Sciences Office (DSO) for designing and developing a
wearable single-eye optical heads-up display (SEHUD) for the military,
particularly infantry soldiers and aircraft pilots. This will address the need
for military infantry soldiers to be equipped with the latest technological
innovations to help them on the field. The main use of the SEHUD will be to
visually feed soldiers real-time data to help their situational awareness. This
can help avoid incidents of friendly fire, inform soldiers of the physical and
human terrain, as well as ease the process of communication.

With the goal of situational awareness in mind, several key design factors
were specified. The SEHUD will be designed according to field of view,
collimation, eyebox, luminance and contrast, boresight, and compatibility
with other communication devices. These design factors will also be used for
product evaluation to see if it has met its objectives.

The SEHUD will be produced under an iterative development cycle. Initially, the
project specifications will be written followed by the product design. This
would involve repeated meetings between engineers and the grantor to make
necessary changes to the specifications. The design will then go under rigorous
analysis, verification, and simulation. Because future design iterations will
be costly, this phase will be vital.

With the design, a rough prototype will be constructed. To construct a
prototype, we plan to order parts such as microprocessors, screens, and sensor
devices from different manufacturers. We will then spend several months writing
drivers to interface with the hardware, and finally we would be able to start
working on high-level features such as overlaying data and voice
communications. After the prototype is finished, it will then be thoroughly
tested with penetration tests and physical tolerance tests to locate bugs and
flaws. The product will finally be evaluated against the design factors and
finally be presented to the grantor.

To develop the SEHUD, we are requesting approximately \$100,000. This includes
salaries, work space, and parts and materials. The allocation of this money is
detailed further within the proposal. We do not believe this to be much
considering the cost of a deployed soldier per year and the potential lives

Monocle Labs is requesting funding to design and develop a single-eye heads-up
display (SEHUD) for use by the military, particularly infantry soldiers. It was
written by Monocle Labs in response to DARPA's DSO's request for proposal to
fund fundamental science, research, and innovation for military defense.

We will first address the need for, and usefulness of, an SEHUD
device for infantry soldiers. The development process for designing the device
will then be outlined. After, it will present the goals and objectives behind
the project as well as the criteria for evaluating the product design. Lastly,
the we will show the project timeline and budgets, itemized and
non-itemized.

Monocle Labs is a tech startup company based in Corvallis, Oregon. We are
devoted to changing the way people interact with computers with a wearable
digital augmented reality. Monocle Labs currently consists of three people, all
of which are co-founders.

Kevin Ngo, our software developer, holds a computer science degree from
Oregon State University. Previously, he has worked at the Oregon State
University Network Engineering Team as a programmer and has worked at the
Mozilla Corporation as a web developer. He prefers working with Linux and
Python but has experience with a handful of languages, frameworks, and database
systems.

Jon Wallace, one of our computer hardware engineers, holds an electrical
and computer engineering degree from Oregon State University. He has experience
in signal processing, discrete communication protocols, and the design
verification processes. His interests are in microprocessor architecture and
design.

Stephen O'brien, one of our other computer hardware engineers, holds an
electrical and computer engineering degree from Oregon State University. He
combines an interest in aesthetic design with the practical knowledge of
computer infrastructure. This combination of viewpoints has led to him
producing interesting and unique past works.

Military infantry soldiers are not equipped with up-to-date technology
which could be used to increase their situational awareness on the field. A
soldier's situational awareness can be the difference between life and death.
Technological advances over the last several decades such as radio
communications and radars have helped increased situational awareness. Though,
there is large room for improvement. Soldier equipment has not been keeping up
with technology which is limiting soldiers' situational awareness. Even end
wearers own more advanced technology in the form of smartphones. As a
result of soldiers not being equipped with recent technology, soldiers
experience the ``fog of war" causing unnecessary difficulty keeping track of
the location of friendly forces, mapping war terrain, and gathering
intelligence.

Soldiers sometimes experience uncertainty in situational awareness during
military operations. Prussian military analyst Carl Von Clausewitz termed this
uncertainty the ``fog of war" and wrote ``war is an area of uncertainty; three
quarters of the things on which all action in War is based on are lying in a
fog of uncertainty to a greater or lesser extent. The first thing (needed) here
is a fine, piercing mind, to feel the truth with the measure of its judgment"
(Regan). Amidst the fog of war, soldiers have trouble determining the
location of friendly forces. This confusion can lead to ``friendly fire", an
inadvertent firing towards one's own or otherwise friendly forces while
attempting to engage enemy forces (Hammer).

During 2006 in Afghanistan, Pfc. Justin R. Davis, 19, was killed as a
result of friendly fire when he was shot in the head by three bullets fired by
U.S. soldiers who said they mistook him for the enemy (Tillman).
This incident was an error in identification where poor terrain and visibility
are major factors. When allied troops are fighting together, the situation can
become even more complex with language barriers to overcome. As shown, friendly
fire often comes as a result of a lack of situational awareness.

There was another situation during 2011 in Pakistan where US troops called
a North Atlantic Treaty Organization (NATO) air attack on the border region
between Afghanistan and Pakistan which mistakingly resulted in the death of
dozens of Pakistani soldiers. This incident unfortunately strained relations
between the US and Pakistan, supposed allies, during tense times of war.
Government officials said the two border posts that were attacked had been
established to try to stop insurgents who use bases in Afghanistan to attack
Pakistan from crossing the border and launching attacks (Boone).
NATO investigators believe that the air strike was meant for those insurgents
(Lemire). This incident is another example of how a lack of
situational awareness can be deadly or cause political strains. Department of
Defense (DOD) officials claimed Pakistani forces fired first, and U.S. troops
didn’t know they were targeting Pakistani soldiers rather than insurgents when
they returned fire (Carroll). Had the troops that called the air
strike were aware that the Pakistani soldiers were allies, the whole situation
may not have happened.

Monocle Labs proposes that with grant funding, we would design and develop
a single-eye heads-up-display (SEHUD) for infantry soldiers. The SEHUD would be
a wearable heads-up display that a soldier could wear over one of their eyes.
It would feature a semi-transparent screen that would overlay information such
as the location of friendly forces and provide information about the physical
and human terrain. It would be able to communicate with other soldiers wearing
SEHUDs through a mesh network. The SEHUD would run an Android platform in which
the military would be able to develop further software applications on top such
as for language translation for communication between allied forces. This
device would provide soldiers with increased situational awareness by feeding
information to the soldiers thus reducing confusion and the effect of the fog
of war. In the words of Clausewitz, the SEHUD would act as the soldier's
fine, piercing mind, to feel the truth with the measure of its judgment".

The goal of the SEHUD system is to improve the situational awareness of
military infantry soldiers. Given more accurate knowledge of the battlefield,
there is a lower risk of friendly fire and an easier accessibility to real-time
intelligence. With these benefits, the effectiveness and precision of military
operations on the ground will dramatically increase. The SEHUD accomplishes
this by visually feeding in data directly to soldiers.

To meet these goals, the SEHUD's system's design will need to sufficiently
address the design factors below. These design factors for heads-up displays
have been well-established through prior research. Using these design factors,
we can quantitatively evaluate the produced SEHUD.

The field of view is the left-to-right range of vision that the wearer has while
wearing the SEHUD. This is an important design feature because it is the
primary motivation for the single-eyed design. Bulkier infantry heads-up
displays have been produced before, but the limited field of view they offer
has decreased their usefulness. Based off of the classic design of the monocle,
which consists of a large lens and narrow frame, the wearer's vision is not
significantly obstructed.

The target field of view of the SEHUD will be a minimum of 110 degrees. The
obstruction caused by the device's frame should be less than 5 degrees and
should remain mainly within the wearer's peripheral vision (Newman).

Collimation is the ability to display information without having to focus the
wearer's eyes away from the environment to process it. This will be challenge
during the design phase since our monocle-based design will consist of a single
pane of glass; the ability to properly collimate the light from the screen with
the light from the outside world will not be trivial but is necessary
(Martin-Emerson).

The eyebox is the range in which the wearer has to be aligned with the display.
It is a byproduct of collimation that plagues most heads-up display systems
Because of the way that collimation is performed, the wearer needs to be aligned
with the screen in a nearly exact position. Otherwise, the information on the
heads-up display will not be visible enough for the wearer to process. Unlike other
heads-up displays, since the SEHUD is designed to be worn very close to the
eye, managing the eyebox will be trivial since it will support a wide range of
motion once the SEHUD is in place. Although proper care must be made to align
the eyebox initially, once this angle has been adjusted, it will not need much
configuration (Newman).

The brightness of the display will need to adjust automatically to the
surrounding environment. With the light-sensing technology already available
today, even in end-user smartphones, this is not difficult to achieve. However,
it is very important to the success of the system. When transitioning between a
brightly-lit outdoor environment to a dark interior environment, the display
should automatically lower its brightness to naturally ease the transition for
the user.

Boresighting is the alignment of information presented on the screen between
the objects that the information is meant to overlay. This process of
alignment is closely related to collimation. Like collimation, boresighting
will be difficult to design because of the angles of the SEHUD's placement
against the wearer's eye and head. These angles require calculation
and interpretation. This will consume a large portion of the SEHUD system's
processing power. Thus, power consumption will have to be accounted for.

Compatibility in this context is the ability to work closely with other
existing communication methods used by military infantry. This would also
communications with intelligence back at bases. The success of the SEHUD
system will depend partly on a fast, simple, and high configurable
information interface.

The section that follows outlines the methods and concepts of the design
process that will be used in the design and prototype construction of the
SEHUD. The design process will have an iterative approach and will consist of
project specification, product design, design verification, product testing,
and evaluation.

Requirements analysis will be used to determine what the SEHUD should do.
This involves collaboration between Monocle Labs and the grantor to agree on
the requirements of the SEHUD system such as the features it needs to have,
the standards it needs to adhere to, or the systems it needs to interface
with. Defining the requirements will require repeated meetings.

From these requirements, engineers will draft specifications to meet those
requirements. Specifications determine how the requirements will be
carried out. Each time the requirements are revised, the specifications will
be revised. The project requirements and specifications process can begin as
soon as the SEHUD project is approved and will continue until the grantor is
satisfied with the project specifications. We expect this to take no longer
than 4 weeks.

Formal meetings will take place remotely with video conference calls. This
will reduce travel costs and quicken the process. We will meet with the
grantor at least twice a week until the requirements and specifications are
well-defined.

The final result of this phase will result in design requirements and
specifications documentation that we can reference during the development of
the SEHUD. Further changes to this document should only happen during the
design verification phase.

During the product design phase, we will design the software and system
architecture. This will consist mainly of determining which hardware and
software components to use as well as drawing hardware and dataflow diagrams.
The actual physical design of the SEHUD will be outsourced to a mechanical
engineering firm that we will select and approve. Below, we will introduce
some design features of the SEHUD itself which the product design will be
have to support.

At the highest level, SEHUD systems will act as nodes to form a mesh network
that will communicate with each other using a close-proximity wireless
network with a range of less than one kilometer. This allows quick data
transfer, which is a natural benefit of a peer-to-peer network. This is
beneficial because information coming from immediately closer nodes worn by
other soldiers has a higher priority than information coming from more
distant sources. By determining the distance from the source of information,
we can prioritize new information to be displayed to the wearer. Up-to-date
and relevant information will be important in maintaining situational
awareness.

Peer-to-peer networks are quick and direct but are cut off from outside
sources of intelligence. This intelligence could be crucial information about
the human terrain that would benefit the infantry. Introducing a few nodes
that have connections to outside sources would readily make this remote
information available to other nodes.

The SEHUD will be designed to be worn over a single eye. It will display
information that will digitally ``augment" what the wearer can currently see.
This method will not require the wearer to direct their attention to a
separate device. Rather, it will unobtrusively enhance their spatial awareness
of threats or objectives. Worn on the non-dominant eye, the SEHUD device
will not affect the wearer's ability to operate machinery, to use firearms,
or to properly see.

The software that runs the SEHUD will be developed simultaneously with the
hardware and will be rapidly deployed and updated. Due to this rapid
deployment, software tests that fail design verification will not trigger
another product design iteration. Instead, the development will follow the
well-known agile Method for software development. Using this method, stable
and testable code can be rapidly written to conform to changing design
requirements and specifications.

The initial design phase is expected to take 16 weeks to complete, resulting
in a thorough design of each part of the SEHUD. Subsequent iterations of the
design phase, as called for during the design verification and product
testing steps, will be given 8 weeks to make necessary changes before moving
to the design verification phase again.

The design verification phase checks that the product design conforms to the
specifications. It determines whether the proposed design does what is
intended. During this process, briefings will be held with the grantor to
determine the value of the current design and evaluate its adherence to the
specification. Any changes to the design will signal the need for an
additional design iteration phase. Note that new features will be difficult
to add at this phase and can significantly delay the project. However if
absolutely necessary, crucial changes can be made to the requirements and
specification.

Formal verification will be done through simulations, peer review, and other
analytical methods. Once the design verification is complete and no new
features are requested, the design then be used for prototype construction.

Once prototype construction begins, the design will be put under a feature
freeze where no new features will be allowed into the design. This reduces
the complexity and the cost of the prototyping phase. There should be no
further features added regardless after completing design verification.

To assemble a prototype, we will order materials, parts, and components such
as microprocessors, chips, sensors, and LCD screens. The cost of these
materials will be detailed in the budgets. Monocle Labs will then begin
constructing a rough-working prototype. The hardware team will connect the
I/O devices such the heads-up display, accelerometer, the GPS, and the
gyroscope onto the bus to the microprocessor. They will then set up the
processor that will be used to process data from the I/O devices. They will
focus on powering the prototype. The prototype will need to be assembled
before software development can begin since the prototype is the programming
environment.

Once a rough initial prototype has been assembled, the software team will
begin to program it. They will program drivers for the I/O devices to
interface with the microprocessor. Information from the I/O device will then
be redirected to the data processing unit which determines what to display on
the heads-up display. The software team will have to write programs to
process the data on the data processing unit as well as work with OpenGL
which a 2D/3D graphics API used to properly display information on the
screen.

The whole process of assembling of the device and writing device drivers and
applications will comprise a large part of the project and will take at least
two months.

Once a working prototype has been built, it will undergo testing. Physical
testing will show that the design will be able to withstand different
environments. Different environments include different physical conditions
such as temperature as well as different electronic conditions such as levels
of noise. Stress testing will be a large part of this process. Subjecting the
network to large loads of erroneous data will ensure that the system is
secure against denial-of-service attacks.

Our software team will be responsible for analyzing the security for the
communication protocols of the SEHUD. This will include verifying the
strength of encryption and protecting against device spoofing. If packets are
not securely encrypted such as through public-key encryption, then unwanted
eavesdroppers can snoop on the communication channels and gain access to what
would then be compromised information. Sending a key will every packet would
prevent spoofing as well.

The hardware team will be responsible for the physical reliability of the
design. They will manage the physical tolerances of the components and ensure
that mass production of the product will be cost-effective. The SEHUD will
have to withstand high temperatures, electronic jamming, and jerking motions.

Any problems found during this stage will be gathered into a list of
modifications, and the product will be sent back to the design stage where
the problems can be addressed and before continuing. When our engineers are
satisfied that the prototype meets all specifications, the final design is
sent to the grantor along with the passing prototype for evaluation.

The evaluation phase will determine whether the SEHUD prototype has
met the requirements and specifications.

We will evaluate whether or not we have met our objectives by
testing the product against the design factors listed in Goals
and Design Factors.

We will sample, test, and measure results in our labs.

After finishing in-house evaluations, we will present the finished
prototype to the grantor.

We are asking for \$ 180000 to design this project over the course of a year.
The grant funding we are requesting will be the primary source of funding for

Soldiers on the battlefield are not equipped with the latest technology which
could be used to help their situational awareness. With grant funding, Monocle
Labs seeks to solve that issue by developing an SEHUD for military purposes.
The SEHUD would provide information about physical and human terrain and thus
keeping soldiers aware of the conditions of the battlefield. With several
different design factors in mind, the SEHUD would be developed from the ground
up, from specification to prototype construction to evaluation.

Technology has been exponentially increasing in recent years, yet soldiers are
still equipped with last century's standard equipment. With the same amount of
costs spent on setting up a radio network on foreign soil, the military could
help develop a mesh network with innovative portable devices instead. More
importantly, the SEHUD should be funded since it could potentially save
hundreds of lives of soldiers. With the cost of deploying each soldier for a
year approaching \$400,000, we should not restrain in equipping them properly
to preserve them. If \$400,000 is the cost of each deployed soldier per year,
what we ask is an extremely small amount if it means equipping every soldier
with an SEHUD.

For us to be able to develop the SEHUD, we need grant funding. As our budgets
detailed, we will need approximately \$100,000 to have a rough prototype
produced in a year. As said earlier, this is but a small cost. The infantry
would be able to operate much more effectively. Even greater, hundreds of
soldiers' lives could potentially be saved. The overhaul of standard-issue
soldier equipment is long overdue.
